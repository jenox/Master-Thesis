\section{Incremental Transformation to Polygonal Dual}
\label{sect:incremental-transformation-to-dual}

\old{The incremental transformation to dual phase is different from the incremental phase discussed previously in that it doesn't produce yet another sequence of operations. Instead, it takes the sequence of operations produced in the previous phase and applies it directly to the embedded boundary graph produced by the non-incremental transformation or optimization phase in an earlier run through the pipeline. This is also the point in the pipeline where it becomes indispensable that the propagated sequence of operations is tailored towards an already-existing product of the pipeline.}

\old{Applying operations for an embedded filtered graph to a boundary graph may appear counter-intuitive or even impossible at first. Keep in mind though that the boundary graph is effectively the dual of the filtered graph and the operations therefore translate 1-to-1 to the boundary graph: operations involving vertices become operations involving faces, edges become face adjacencies, internal faces become points where multiple countries meet, and the outer face becomes the faces that have borders with nothing on the other side. We must therefore be able to make the following changes to an embedded boundary graph:}

\begin{itemize}
	\setlength\itemsep{-0.25em}
	\item \old{insert a region on the inside of the map}
	\item \old{insert a region on the outside of the map}
	\item \old{remove a region on the inside of the map}
	\item \old{remove a region on the outside of the map}
	\item \old{flip a region adjacency on the inside of the map}
	\item \old{create an adjacency between two regions on the outside of the map}
	\item \old{remove an adjacency between two regions on the outside of the map}
	\item \old{update a region's or a region adjacency's weight}
\end{itemize}

\old{These changes have the same preconditions as their counterparts on the embedded filtered graph. We will provide visualizations and concrete implementations for making theses in \cref{chap:implementation}.}

\subsection{Insert Vertex Inside}
\subsection{Insert Vertex Outside}
\subsection{Remove Internal Vertex}
\subsection{Remove External Vertex}
\subsection{Flip Internal Edge}
\subsection{Insert Edge Outside}
\subsection{Remove Edge Outside}
