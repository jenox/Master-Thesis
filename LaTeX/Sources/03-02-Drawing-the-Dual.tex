\section{Drawing the Polygonal Dual}
\label{sect:drawing-the-dual}

In the second step of the pipeline, we apply a force-directed graph drawing algorithm to the initial map graph $G_\text{init}$ produced by \cref{alg:transformation-to-dual} to generate the approximately area-proportional map graph $G_\text{prop}$.

In a force-directed algorithm, we interpret a graph's vertices as particles in a physical system. Based on the structure of the graph and the relative position of the particles, we define define various forces that act to bring the system to a stable equilibrium position in which its potential energy is at a local minimum. In these equilibrium positions, the system is in a somewhat relaxed state that, in the context of graph drawing, generally is a visually appealing drawing of the graph. We find an equilibrium position by iteratively computing the net force acting on each particle and displacing it by a small amount in the direction of the net force, based on its absolute value.

\tamara{Can/should I add more theory on the physical aspect here?}



\paragraph{Forces}

We define the forces that exist in our particle system with two goals in mind: First, the faces of the graph (the regions of the map) to have an area ought to be close to proportional to some prescribed value. And second, we want the faces to be \emph{locally fat}. Our intuitive understanding of local fatness is that a region shouldn't have drawn-out, tight corridors. We formalize and discuss different quantifiable measures that aim to capture the local fatness of regions in \cref{chap:evaluation}. Note that both of these are nice-to-have goals though. Planarity of the resulting drawing, on the other hand, is a hard requirement that must be preserved at all costs.

Let us now discuss the concrete force components that act to bring our particle system to an equilibrium position.
\tamara{Do you have a better idea than putting this in a bulleted list? Do we want figures for pressure and angle?}
%
\begin{itemize}
	\item \textbf{Air Pressure:}
	First, we define forces based on the air pressure in the faces as suggested by Alam \etal{} \cite{alam2013computing}. They are responsible for growing faces that are currently compressed and shrinking faces that are currently larger than they should be, therefore working towards the statistical accuracy of the generated map. We want this force and the overall drawing to be agnostic to constant factors of the face weights and therefore compute the normalized pressure $P(f)$ in a face $f$ as
	\begin{equation*}
		P(f) \coloneqq \frac{w(f)}{A(f)} \cdot \frac{\sum\limits_{g \in F}{A(g)}}{\sum\limits_{g \in F}{w(g)}},
	\end{equation*}
	where $A(f)$ is the area currently covered by some face $f$ and $w(f)$ its weight. A value greater than $1$ indicates that a regions wants to grow, a value less than $1$ indicates that it wants to shrink. We define the exerted force to be logarithmic in $P(f)$ such that the sign of $\log(P(f))$ determines whether the force points outwards or inwards and distribute the force proportionally along all edges bounding the face:
	\begin{equation}
		\vec{F}_P(f,e) \coloneqq 10 \cdot \log(P(f)) \cdot \frac{l(e)}{l(f)} \cdot \vec{n}_e
	\end{equation}
	Here, $l(e)$ is the length of the edge $e$, $l(f)$ is the length of the boundary of the face $f$, as $\vec{n}_e$ is the outward-pointing normal vector of the edge $e$. We apply $\vec{F}_P(f,e)$ to both endpoints of $e$ for all pairs $(f,e)$ of internal faces and edges where $e$ is incident to $f$.

	\item \textbf{Internal Angles:}
	Internal face angles close to $0^\circ$ cause tight corridors in the form of pointy spikes and angles close to $360^\circ$ the opposite \emdash{} both features we want to avoid if possible. We therefore define a force that works against such extreme angles, attempting to create uniform internal angles in each of the individual faces. Given a face $f$ on $n$ vertices, we define $\measuredangle_f$ as the internal angle of a regular $n$-gon. For all pairs $(v,f)$ of vertices and faces where $v$ lies on $f$'s boundary, we define the force
	\begin{equation}
		\vec{F}_\measuredangle(f,v) \coloneqq 1 \cdot \left\{\begin{array}{lr}
			\frac{\measuredangle_v - \measuredangle_f}{360^\circ - \measuredangle_f}, & \text{if } \measuredangle_v \geq \measuredangle_f\\
			\frac{\measuredangle_v - \measuredangle_f}{\measuredangle_f}, & \text{otherwise}\end{array}
		\right\} \cdot \vec{n}_v.
	\end{equation}
	Here, $\measuredangle_v$ is the internal angle at vertex $v$ in face $f$ and $\vec{n}_v$ is the normalized outward-pointing bisecting vector at $v$. We apply this force to all vertices on a face's boundary.

	\item \textbf{Vertex-vertex repulsion:}
	We define a repulsive force between pairs of vertices to prevent the vertices from clumping together. This is important because in order to preserve planarity, vertices that are very close to others will need to have their movement restricted severely, possibly hindering us from satisfying our aesthetic criteria. We think of the vertices as charged particles that push each other away and define a repulsive force based on Coulomb's law that is exerted along the line connecting pairs of vertices and whose magnitude depends on their Euclidean distance $d(u,v)$:
	\begin{equation}
		F_\leftrightarrow(u,v) \coloneqq 25 \cdot \frac{1}{d(u,v)^2}
	\end{equation}
	For every pair of vertices $(u, v) \in V^2, u \neq v$, we apply this force to both $u$ and $v$, though in opposite directions.

	\item \textbf{Vertex-edge repulsion:}
	In another attempt to prevent tight corridors from forming, we define an additional repulsive force between vertices and edges. Given an edge $e$ and non-incident vertex $v$, we define $d(e,v)$ as the shortest Euclidean distance between $v$'s position and any point on $e$. Analogous to the vertex-vertex-repulsion, we define the force as an inverse of the squared distance:
	\begin{equation}
		F_\bot(e,v) \coloneqq 10 \cdot \frac{1}{d(e,v)^2}
	\end{equation}
	For all pairs $(e,v)$ of non-incident edges and vertices that share a boundary of a face, we apply a force with magnitude $F_\bot(e,v)$ to $v$. The force is applied perpendicular to $e$ and oriented to the side of $e$ on which $v$ lies.
\end{itemize}

The constant factors of the force components above were determined experimentally.

Note that we do not define attractive forces between adjacent vertices as most traditional force-directed algorithms do. Such an attractive force is generally used to keep adjacent vertices close together while nonadjacent vertices are pushed further apart by the repulsive forces. This works for many graph drawing algorithms because for them, uniform edge length is a desired feature in equilibrium. In our case, however, there is no correlation between the extent of a region and the number of edges on its boundary and, subsequently, the length of the edges on its boundary. Including such an attractive force here would in fact be counterproductive.

To make sure that the physical simulation converges, we cool the particle system down over time. We define a global cooling parameter $\alpha = 0.01$ and, at step $i$ of the iterative process, add $(1 - \alpha)^i$ as a factor to all of the forces mentioned above. By doing so, we prevent unstable oscillations around equilibrium.



\paragraph{Preventing Edge Crossings}

When iteratively displacing the map graph's vertices according to the forces defined above, we must pay close attention to not accidentally introduce edge crossings.

To do so, we adopt the rules of PrEd \cite{bertault1999force} that ensure no edge crossings are created. At each step of the algorithm, PrEd computes the maximum distance each of the vertices is allowed to move such that the drawing's edge crossing properties are guaranteed to be preserved. The maximum displacement per vertex is computed in eight general directions, zones of $45^\circ$ each. The actual displacement of the vertices is then clamped at the maximum distance the vertex can safely move in the desired direction.
