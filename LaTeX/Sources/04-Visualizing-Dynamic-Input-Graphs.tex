\chapter{Visualizing Dynamic Input Graphs}
\label{chap:visualizing-dynamic-input-graphs}

Extending the approach discussed in the previous section to dynamic input graphs is challenging primarily because we must try to preserve the viewer's mental map as the underlying data changes over time.
We want the visualization at different points in time to be similar enough so that the viewer can clearly tell what parts have changed \cite{mashima2011visualizing}, yet allow for the required changes in geography and topology.
Still, changes between visualizations at consecutive points in time should minimize movement and allow for smooth animations therebetween.

The pipeline for static inputs discussed in the previous section does not satisfy these requirements.
Running through the entire pipeline with a different, albeit similar, input graph, may result in a completely different visualization, destroying the viewer's mental map.
We therefore extend the pipeline in a way that allows for small, incremental changes to be propagated through the pipeline and to eventually be applied to the previous output in a way that preserves the viewer's mental map.

We extend the pipeline for static input by an incremental transformation phase.
This phase takes two inputs: A proportional map graph $G_\text{prop}$ that the pipeline previously produced as output for some cluster graph $G_\text{emb}$, and a sequence of operations on said cluster graph, that, when applied to $G_\text{emb}$, yields the cluster graph $G_\text{emb}^\prime$.
The incremental embedding phase then determines how these operations translate to a polygonal dual of $G_\text{emb}$ and applies the translated operations to $G_\text{prop}$, producing $G_\text{init}^\prime$, a (not necessarily approximately area-proportional) polygonal dual of $G_\text{emb}^\prime$.
This polygonal dual is then fed back into the drawing phase to make it approximately area-proportional and improve the local fatness of the map's regions.

\begin{figure}[H]
	\centering\includegraphics[width=0.9\textwidth]{Resources/Pipeline-Thesis-Dynamic.pdf}
	\caption{Overview of the algorithmic pipeline for dynamic input graphs.}
	\label{fig:dynamic-pipeline-thesis}
\end{figure}

Real-world applications such as visualizing a dynamic opinion network need a way to feed a sequence of operations on the embedded cluster graph into our framework.
This could be done by prepending an incremental clustering phase that translates changes to the simple input graph into changes of its embedded cluster graph.
However, such a sequence of operations is only meaningful in combination with a graph that these operations can be applied to.
One must provide the previously-produced cluster graph as additional input to the incremental clustering phase such that it can tailor its output to the cluster graph that has already been locked in in an earlier run through the pipeline.

This tweak to our pipeline is illustrated in the following figure:
%
\begin{figure}[H]
	\centering\includegraphics[width=0.9\textwidth]{Resources/Pipeline-Application-Dynamic.pdf}
	\caption{Overview of a possible algorithmic pipeline for generic applications.}
	\label{fig:dynamic-pipeline-application}
\end{figure}

Extending the pipeline to allow the propagation of small, incremental changes of the input graph has numerous benefits other than the ability to preserve the viewer's mental map:
%
\begin{itemize}
\item It allows highly efficient implementations of the incremental parts of the pipeline as only the aspects that have actually changed in the input graph or intermediate products need to be processed and propagated further along the pipeline.
\item It makes the dynamic pipeline highly parallelizable: when a later phase is processing changes, an earlier phase can already start processing new changes independently.
With our force-directed implementation of the drawing phase, we can even incorporate dynamic updates while the drawing phase is still running, even if it has not converged yet: we pause the force simulation, feed the current map graph $G_\text{prop}$ into the incremental transformation phase to incorporate the dynamic updates, and then resume the simulation with the updated map graph $G_\text{init}^\prime$ produced by the incremental transformation phase.
\item It efficiently supports dynamic input in an online setting, \ie{} a setting in which the incremental changes aren't known in advance, for example when visualizing live data.
\end{itemize}



\paragraph{Supported Operations}

Our pipeline supports numerous classes of primitive operations on the cluster graph, such as inserting and removing vertices and edges, flipping edges, or simply changing a cluster's weight.
By composing multiple primitive operations in a sequence, more drastic changes can be made to the cluster graph.
In our pipeline, the operations are applied one by one nonetheless.

The simplest operation of all is changing a vertex $v$'s weight: We simply take the previous proportional map graph $G_\text{prop}$, update the weight of the face $f_v$ corresponding to the vertex $v$, and declare that as the new initial map graph $G_\text{init}^\prime$.
$G_\text{init}^\prime$ then runs through the drawing phase again to account for the updated face weights.

Implementing the remaining operations as part of the incremental transformation is a little more challenging, and we'll discuss those in great detail in the following sections.

\clearpage
\section{Inserting Vertices}
\label{sect:inserting-vertices}

When a new cluster appears in our underlying data set, we want to add a new vertex to the cluster graph. We distinguish between adding a new vertex on the inside and adding a new vertex on the outside because different rules apply.



\paragraph{Inserting Vertices Inside}

First, let us discuss adding a vertex on the inside. All internal faces of the cluster graph are triangles. If we add a vertex in one of the triangular faces, we must also add edges to the three vertices bounding the face without introducing edge crossings in order to preserve the graph's internal triangulatedness. A valid vertex insertion is illustrated in \cref{}.

\begin{figure}[H]
	\centering
	\includegraphics[height=30mm]{Resources/InsertVertexInside.png}
	\caption{A cluster graph and a polygonal dual thereof, before (a, b) and after (c, d) inserting the vertex $x$ in the triangular face $uvw$.}
	\label{fig:insert-vertex-inside-example}
\end{figure}



Let $u$, $v$, and $w$ be the vertices bounding an internal face and $x$ the new vertex we want to add inside said face.

\begin{itemize}
	\item compute three incident boundaries ($u$-$v$, $v$-$w$, $w$-$u$)
	\item on all boundaries, pick the subdivision vertex closest (graph-theoretic) to vertex where all three corresponding faces meet. or create it on midpoint if boundary is only a single edge.
	\item we want to add edges between these subdivision vertices and remove their connections to meeting point instead.
	\item we may need a bend per edge! for each edge, distinguish two cases based on angle at meeting point
	\item if >180, just use meeting as bend loc. otherwise do binary search on segment from meeting point to midpoint of subdivs, starting at far end.
\end{itemize}


\paragraph{Inserting Vertices Outside}

Alternatively, we can add a new vertex in the outer face of the cluster graph. Such a vertex must be connected to at least two vertices on the outer face to preserve the graph's 2-connectivity and its neighbors must form a path on the original boundary of the cluster graph in order not to create holes and thereby violate its internal triangulatedness.

We restrict ourselves to adding new vertices in the outer face that are made incident to exactly 2 neighboring vertices. Let $\{u,v\}$ be an edge on the outer face, then we support adding a new vertex $w$ in the outer face and connecting it to both $u$ and $v$.

\begin{figure}[H]
	\centering
	\includegraphics[height=30mm]{Resources/InsertVertexOutside.png}
	\caption{A cluster graph and a polygonal dual thereof, before (a, b) and after (c, d) inserting the vertex $x$ on the outer face and connecting it to $u$ and $v$.}
	\label{fig:insert-vertex-outside-example}
\end{figure}

\begin{itemize}
	\item compute boundaries of $u$ and $v$ with the outer face
	\item on both sides, pick the subdivision vertex closest (graph-theoretic) to the vertex where $u$, $v$, and the outer face meet. or create it on midpoint if a boundary is only single edge.
	\item we want to add edge between those two subdivision vertices. may need to add a bend.
	\item need to distinguish two cases based on external angle at meeting point.
	\item in both cases: binary search for bend location on segment from meeting point to midpoint of segment connecting subdivision vertices.
\end{itemize}

\clearpage
\section{Removing Vertices}
\label{sect:removing-vertices}

same distinguishment: internal vs external, reverses the insert inside/outside.



\paragraph{Removing Internal Vertices}

can only remove internal if degree 3. if higher degree we would create hole. need to flip edges first.\cref{sect:flipping-edges}

\begin{figure}[H]
	\centering
	\includegraphics[height=30mm]{Resources/RemoveInternalVertex.png}
	\caption{A cluster graph and a polygonal dual thereof, before (a, b) and after (c, d) removing the internal vertex $x$ with degree 3.}
	\label{fig:remove-internal-vertex-example}
\end{figure}

\begin{itemize}
	\item compute boundary (path) with the three incident faces: $u$-$x$, $v$-$x$, $w$-$x$
	\item contract these boundaries into single-edge-boundaries (why is this possible?)
	\item we then have a plain triangle
	\item replace triangle with single point in its barycenter
\end{itemize}

 % either hack, assign area to another face by removing one of three boundaries. or: contract boundaries until they are 1 long each -> face is triangle. then place new vertex in barycenter + connect.



\paragraph{Removing External Vertices}

when removing vertices on the outer face along with its incident edges, we must ensure that the graph remains 2-connected afterwards.

as long as that's given, removing such a vertex is easy though. we just remove the respective face's boundary with the outer face.

\begin{figure}[H]
	\centering
	\includegraphics[height=30mm]{Resources/RemoveInternalVertex.png}
	\caption{A cluster graph and a polygonal dual thereof, before (a, b) and after (c, d) removing the vertex $x$ on the outer face.}
	\label{fig:remove-external-vertex-example}
\end{figure}

\begin{itemize}
	\item find boundary of respective face with outer face
	\item remove all internal/subdivision vertices and edges on this path
\end{itemize}

\clearpage
\section{Flipping Edges}
\label{sect:flipping-edges}

Let us now discuss the edge flip mentioned in the previous sections.
An internal edge $\{u,v\}$ is incident to two different internal faces $f$, $g$.
Let $x$ and $y$ denote the third vertex bounding $f$ and $g$, respectively.
It is $x \neq y$ because the cluster graph is simple.
Flipping the edge $\{u,v\}$ would replace it with the edge $\{x,y\}$.
Consequently, this operation is only permitted iff $x$ and $y$ are not already adjacent \emdash{} otherwise, we would introduce a duplicate adjacency.
\Cref{fig:flip-edge-example-internal} shows an example of a valid edge flip operation.

\begin{figure}[H]
	\centering
	\subfigure[]{\includegraphics[height=29mm]{Resources/FlipEdge-Example-Internal-1.pdf}}
	\quad
	\subfigure[]{\includegraphics[height=29mm]{Resources/FlipEdge-Example-Internal-2.pdf}}
	\qquad
	\subfigure[]{\includegraphics[height=29mm]{Resources/FlipEdge-Example-Internal-3.pdf}}
	\quad
	\subfigure[]{\includegraphics[height=29mm]{Resources/FlipEdge-Example-Internal-4.pdf}}
	\caption{A cluster graph and a polygonal dual thereof, before (a, c) and after (b, d) flipping the internal edge $\{u,v\}$.}
	\label{fig:flip-edge-example-internal}
\end{figure}

An edge flip in a cluster graph translates to region adjacencies being flipped in its dual.
Given a polygonal dual of some cluster graph, we apply an edge flip in two phases.
First, we contract the region boundary we want to remove into a single point, creating a degenerate contact representation in which four regions meet in a point.
In the second phase, we create a region boundary in the opposite direction, getting rid of the degeneracy at the point into which the original boundary has been contracted.

Let $u$ and $v$ be two adjacent faces in the polygonal dual whose boundary we want to contract.
Also, let path $P_{uv}$ be the maximal common boundary between $u$ and $v$, oriented such that $u$ lies on the left of it, and $v$ lies on the right of it.
At both endpoints of the path, $u$ and $v$ meet with a third face.
We denote the third face incident to the path's first vertex by $x$ and the third face incident to the path's last vertex by $y$, as illustrated in \cref{fig:flip-edge-example-internal}.
To contract the $u$-$v$-boundary into a single point, we repeatedly contract a peripheral edge on the boundary until the last edge has been contracted.
We do so on alternating ends, \ie{}, we start by contracting the first edge, then the last, then the first again, etc.

\begin{figure}[H]
	\centering
	\subfigure[]{\includegraphics[width=40mm]{Resources/FlipEdge-ContractBoundaryBelow-1.pdf}\label{subfig:flip-edge-contract-boundary-below-1}}
	\quad
	\subfigure[]{\includegraphics[width=40mm]{Resources/FlipEdge-ContractBoundaryBelow-2.pdf}\label{subfig:flip-edge-contract-boundary-below-2}}
	\quad
	\subfigure[]{\includegraphics[width=40mm]{Resources/FlipEdge-ContractBoundaryBelow-3.pdf}\label{subfig:flip-edge-contract-boundary-below-3}}
	\caption{A contact representation before (a) and after (c) contracting the peripheral edge $\{p_{uv},p_{uvy}\}$ on the $u$-$v$-boundary away from $y$. (b) shows the construction of potential subdivision vertices.}
	\label{fig:flip-edge-contract-boundary-below}
\end{figure}

\Cref{fig:flip-edge-contract-boundary-below} illustrates how we contract the first edge of the oriented upwards without introducing edge crossings.
We describe only this construction in writing; the contraction of the last edge works virtually the same way and is illustrated in \cref{fig:flip-edge-contract-boundary-above}.
Let $p_{uvy}$ denote the vertex where the faces $u$, $v$, and $y$ meet and $p_{uy}$ and $p_{vy}$ the subdivision vertices on the $u$-$y$- and $v$-$y$-boundaries that are incident to $p_{uvy}$, respectively.
If the $u$-$y$- or $v$-$y$-boundary consists of only one edge, we subdivide it at its midpoint first.
Let $p_{uv}$ be the subdivision vertex on the $u$-$v$-boundary that is incident to $p_{uvy}$ or the last vertex of the oriented boundary if no such subdivision vertex exists.
To reduce the length of the $u$-$v$-boundary by one, we would want to remove $p_{uvy}$ and its incident edges and add edges from $p_{uv}$ to both $p_{uy}$ and $p_{vy}$.
These edges may introduce crossings, though, as illustrated by the dashed lines in \cref{subfig:flip-edge-contract-boundary-above-2} and \cref{subfig:flip-edge-contract-boundary-below-2}.
However, with just one bend on each of the edges, we can guarantee that no edge crossings are created:

\begin{itemize}
\item If adding the edge between $p_{uv}$ and $p_{ay}$ ($a \in \{u,v\}$) does not introduce a crossing, we simply add the edge.
(for $a = u$ in \cref{subfig:flip-edge-contract-boundary-below-2})
\item Otherwise, if the internal angle of face $a$ at $p_{uvy}$ is $180^\circ$ or more, we place the bend at $p_{uvy}$, \ie{}, we insert the edge $\{p_{uv},p_{ay}\}$ and subdivide it with a new vertex $q_{ay}$ at the position of $p_{uvy}$.
Note that at most one of the faces can have an internal angle at $p_{uvy}$ that is $180^\circ$ or more.
(for $a = v$ in \cref{subfig:flip-edge-contract-boundary-above-2})
\item Otherwise, we search for a bend location in the form of a subdivision vertex $q_{ay}$ somewhere on the outward-pointing bisector of the angle $\angle_{p_{ay}p_{uvy}p_{uv}}$ (dotted lines in \cref{subfig:flip-edge-contract-boundary-below-2} and \cref{subfig:flip-edge-contract-boundary-above-2}).
We start looking at the point where the bisector intersects the segment from $p_{uv}$ to $p_{ay}$ and repeatedly divide the remaining distance to $p_{uvy}$ in half until we find a bend location for which the bent edge from $p_{uv}$ to $p_{ay}$ would not introduce edge crossings.
As the candidate location moves infinitesimally close to $p_{uvy}$, we are guaranteed to find one that does not introduce crossings.
(for $a = v$ in \cref{subfig:flip-edge-contract-boundary-below-3} and $a = u$ in \cref{subfig:flip-edge-contract-boundary-above-3})
\end{itemize}

\begin{figure}[H]
	\centering
	\subfigure[]{\includegraphics[width=40mm]{Resources/FlipEdge-ContractBoundaryAbove-1.pdf}\label{subfig:flip-edge-contract-boundary-above-1}}
	\quad
	\subfigure[]{\includegraphics[width=40mm]{Resources/FlipEdge-ContractBoundaryAbove-2.pdf}\label{subfig:flip-edge-contract-boundary-above-2}}
	\quad
	\subfigure[]{\includegraphics[width=40mm]{Resources/FlipEdge-ContractBoundaryAbove-3.pdf}\label{subfig:flip-edge-contract-boundary-above-3}}
	\caption{A contact representation before (a) and after (c) contracting the last remaining edge $\{p_{uvx},p_{uvy}\}$ on the $u$-$v$-boundary away from $y$. (b) shows the construction of potential subdivision vertices.}
	\label{fig:flip-edge-contract-boundary-above}
\end{figure}

Once the $u$-$v$-boundary has been contracted into a single vertex $p_{uvxy}$ where all four faces $u$, $v$, $x$, and $y$ now meet, as shown in \cref{subfig:flip-edge-contract-boundary-above-3}, we need to resolve the degeneracy and create a boundary in the opposite direction, \ie{}, an $x$-$y$-boundary.
Let $p_{ux}$, $p_{uy}$, $p_{vx}$, and $p_{vy}$ denote the subdivision vertices on the respective boundaries that are incident to $p_{uvxy}$.
Again, if a boundary consists of only one edge, we subdivide it at its midpoint first such that all $p_{ab}$ exist.
We are now looking to stretch the vertex $p_{uvxy}$ back into a non-degenerate $x$-$y$-boundary, as illustrated in \cref{fig:flip-edge-create-boundary}.
Analogous to above, we search for a position in face $u$ ($v$) where we can place a vertex $q_u$ ($q_v$) and add edges to $p_{uvxy}$, $p_{ux}$, and $p_{uy}$ ($p_{uvxy}$, $p_{vx}$, and $p_{vy}$) without introducing crossings.
We start at the intersection of the segment between $p_{ux}$ and $p_{uy}$ (between $p_{vx}$ and $p_{vy}$) and the bisector of $\angle_{p_{ux}p_{uvxy}p_{uy}}$ ($\angle_{p_{vx}p_{uvxy}p_{vy}}$) and repeatedly divide the distance to $p_{uvxy}$ in half until we find a valid position.
Once we have found such a position, we insert the vertex $q_u$ ($q_v$) along with the aforementioned edges and remove the edges $\{p_{uvxy},p_{ux}\}$ and $\{p_{uvxy},p_{uy}\}$ ($\{p_{uvxy},p_{vx}\}$ and $\{p_{uvxy},p_{vy}\}$).
In case $\measuredangle_{p_{ux}p_{uvxy}p_{uy}} \geq 180^\circ$ ($\measuredangle_{p_{vx}p_{uvxy}p_{vy}} \geq 180^\circ$), we will not find such a position and shall not make an adjustment on that side.
However, again, this event cannot occur for both $u$ and $v$ at the same time.
As a result, we always create an $x$-$y$-boundary that consists of at least one edge.

\begin{figure}[H]
	\centering
	\subfigure[]{\includegraphics[width=45mm]{Resources/FlipEdge-StretchBoundary-1.pdf}}
	\quad
	\subfigure[]{\includegraphics[width=45mm]{Resources/FlipEdge-StretchBoundary-2.pdf}}
	\quad
	\subfigure[]{\includegraphics[width=45mm]{Resources/FlipEdge-StretchBoundary-3.pdf}}
	\caption{A contact representation before (a) and after (c) creating a non-degenerate yellow-blue adjacency. (b) shows the construction of potential subdivision vertices.}
	\label{fig:flip-edge-create-boundary}
\end{figure}



\paragraph{Inserting and Removing Edges}

As clusters in our data set grow increasingly similar, we may want to indicate this similarity with new edges between existing clusters in the cluster graph.
Similarly, clusters can grow apart, and we may want to remove edges between clusters in the cluster graph.
However, because the filtered cluster graph is internally triangulated, we cannot insert any more edges on the inside of the graph.
Removing an internal edge is not permitted either, as that would create a hole in the graph.
Consequently, we can insert edges only in the outer face and remove edges only on the outer face.

On top of that, inserting an edge in the outer face is only possible if it preserves the cluster graph's internal triangulatedness.
Therefore, inserting an edge $\{u,w\}$ is only permitted iff $u$ and $w$ lie on the outer face and have a neighbor $v$ in common that also lies on the outer face.
The inserted edge is then embedded such that it forms a new triangular internal face with $v$ and turns $v$ into an internal vertex.
If exactly four vertices bound the outer face before inserting the edge $\{u,w\}$, there exist two candidates for $v$.
In this case, it must be made explicit which one is supposed to become internal and which one is supposed to remain on the outer face.
\Cref{fig:flip-edge-example-insert} shows an example of a valid edge insertion.

\begin{figure}[H]
	\centering
	\subfigure[]{\includegraphics[height=29mm]{Resources/FlipEdge-Example-Insert-1.pdf}}
	\quad
	\subfigure[]{\includegraphics[height=29mm]{Resources/FlipEdge-Example-Insert-2.pdf}}
	\qquad
	\subfigure[]{\includegraphics[height=29mm]{Resources/FlipEdge-Example-Insert-3.pdf}}
	\quad
	\subfigure[]{\includegraphics[height=29mm]{Resources/FlipEdge-Example-Insert-4.pdf}}
	\caption{A cluster graph and a polygonal dual thereof, before (a, c) and after (b, d) inserting the edge $\{u,w\}$ to form an internal triangular face with $v$.}
	\label{fig:flip-edge-example-insert}
\end{figure}

Removing an edge $\{u,w\}$ on the outer face of the cluster graph is only permitted if the graph remains biconnected.
This property is preserved iff both $u$ and $w$ have degree $d(\cdot) \geq 3$ and the third vertex $v$ in the internal face bounded by $\{u,w\}$ does not already lie on the outer face.
If that vertex laid on the outer face already, we would end up creating a duplicate adjacency/boundary between $v$ and the outer face, which we specifically excluded in \cref{chap:preliminaries}.
\Cref{fig:flip-edge-example-remove} shows an example of a valid edge removal.

\begin{figure}[H]
	\centering
	\subfigure[]{\includegraphics[height=29mm]{Resources/FlipEdge-Example-Remove-1.pdf}}
	\quad
	\subfigure[]{\includegraphics[height=29mm]{Resources/FlipEdge-Example-Remove-2.pdf}}
	\qquad
	\subfigure[]{\includegraphics[height=29mm]{Resources/FlipEdge-Example-Remove-3.pdf}}
	\quad
	\subfigure[]{\includegraphics[height=29mm]{Resources/FlipEdge-Example-Remove-4.pdf}}
	\caption{A cluster graph and a polygonal dual thereof, before (a, c) and after (b, d) removing the edge $\{u,w\}$.}
	\label{fig:flip-edge-example-remove}
\end{figure}

Both of these operations are again a special case of the generic edge flip discussed above with a quirk:
Instead of four internal faces, they involve three internal faces and the implicit outer face.

In the example from \cref{fig:flip-edge-example-insert}, we make $v$ into an internal vertex by inserting the edge $\{u,w\}$, removing $v$'s boundary with the outer face in the dual and creating a $u$-$w$-boundary in turn.
Again, recall that in the construction of the augmented dual from \cref{def:augmented-dual}, we insert a helper vertex $v^+$ in the outer face and add edges to all vertices on the original outer face.
This helper vertex and its adjacencies correspond to the outer face and its boundaries in the dual.
When inserting an edge in the outer face, we are essentially flipping the helper edge $\{v,v^+\}$ to become the inserted edge $\{u,w\}$.
Therefore, we apply the same procedure as discussed for the edge flip above, contracting the boundary between $v$ and the outer face into a single point and then creating a non-degenerate $u$-$w$-boundary.

Similarly, when removing an edge $\{u,w\}$ on the outer face as in \cref{fig:flip-edge-example-remove}, a previously-internal vertex $v$ moves onto the outer face.
We can think of it as flipping the edge to be removed to become the helper edge $\{v,v^+\}$.
In the polygonal dual, this gets rid of the $u$-$w$-boundary and creates a boundary between $v$ and the implicit outer face in turn.
Again, we apply the same procedure as discussed for the edge flip, first contracting the $u$-$w$-boundary into a single point and then creating a non-degenerate boundary between $v$ and the outer face.

