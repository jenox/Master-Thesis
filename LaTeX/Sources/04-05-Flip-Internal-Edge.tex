\section{Flipping Edges}
\label{sect:fliping-edges}

Let us now look at operations involving only edges. We cannot add new edges inside, because the cluster graph is already internally triangulated. And we can't remove internal edges either, because that would create a hole and violate the internal triangulatedness. However, we can flip internal edges:

An internal edge $\{u,v\}$ is incident to two different internal faces $f$, $g$. Let $x$ and $y$ denote the third vertex bounding $f$ and $g$, respectively. It is $x \neq y$ because the cluster graph is simple. Flipping the edge $\{u,v\}$ would replace it with the edge $\{x,y\}$. This operation is therefore only permitted if $x$ and $y$ are not already adjacent \emdash{} otherwise we would introduce a duplicate adjacency. A valid edge flip operation is illustrated in \cref{fig:flip-internal-edge-example}.

\begin{figure}[H]
	\centering
	\subfigure[]{\includegraphics[width=30mm]{Resources/FlipInternalEdge-Example-1.pdf}}
	\quad
	\subfigure[]{\includegraphics[width=30mm]{Resources/FlipInternalEdge-Example-2.pdf}}
	\quad
	\subfigure[]{\includegraphics[width=30mm]{Resources/FlipInternalEdge-Example-3.pdf}}
	\quad
	\subfigure[]{\includegraphics[width=30mm]{Resources/FlipInternalEdge-Example-4.pdf}}
	\caption{A cluster graph and a polygonal dual thereof, before (a, b) and after (c, d) flipping the internal edge $\{u,v\}$.}
	\label{fig:flip-internal-edge-example}
\end{figure}

An edge flip in a cluster graph translates to region adjacencies being flipped in its dual. Given a polygonal dual of some cluster graph, we apply an edge flip in two phases: First, we contract the respective region boundary into a single point, creating a degenerate contact representation in which 4 regions meet in a point. In the second phase, we create a new boundary in the opposite direction around that point the original boundary has been contracted into.

We contract a region boundary by repeatedly removing a peripheral edge of the boundary on alternating ends until the last edge of the boundary has been removed. To remove a peripheral edge of a boundary, let us consider the general case illustrated in \cref{fig:flip-internal-edge-contract}. We want to get rid of the edge $\{p_1,p_2\}$ on the red-green boundary. To do so, we  remove $p_2$ and its incident edges and add the new edges $\{p_3,p_1\}$ and $\{p_4,p_1\}$. This may introduce edge crossings though, as illustrated in \cref{subfig:flip-internal-edge-contract-crossing}. We may therefore need to introduce a bend to both new edges in the form of a subdivision vertex. We look for possible bend locations on the segment from $p_2$ to $p_3$ and $p_2$ to $p_4$, respectively, using binary search: We start at the far end, and cut the remaining distance to $p_2$ in half until we find a bend location that doesn't introduce edge crossings. As the distance from $p_2$ grows infinitesimally small, we are guaranteed to find a valid bend location.

\begin{figure}[H]
	\centering
	\subfigure[]{\includegraphics[width=40mm]{Resources/FlipInternalEdge-Contract-1.pdf}}
	\quad
	\subfigure[]{\includegraphics[width=40mm]{Resources/FlipInternalEdge-Contract-2.pdf}\label{subfig:flip-internal-edge-contract-crossing}}
	\quad
	\subfigure[]{\includegraphics[width=40mm]{Resources/FlipInternalEdge-Contract-3.pdf}}
	\caption{A contact representation before (a) and after (c) removing the peripheral edge $\{p_1,p_2\}$ from the red-green boundary. (b) shows that we require a subdivision vertex $q_1$ on the red-blue boundary.}
	\label{fig:flip-internal-edge-contract}
\end{figure}

Once the boundary has been contracted into a single point, we need to resolve the degeneracy and create a boundary in the opposite direction as illustrated in \cref{fig:flip-internal-edge-expand}. If $\measuredangle_{p_1p_3p_4} < 180^\circ$, we search for a position in the face bounded by $p_1$, $p_3$, and $p_4$ at which we can place a new vertex $q_1$ and connect it to those three vertices without introducing edge crossings. We do this using another binary search on the segment from $p_3$ to the midpoint of $p_1$ and $p_4$. Once we found a valid position for $q_1$ and have added it to the graph along with the aforementioned edges, we remove the edges $\{p_3,p_1\}$ and $\{p_3,p_4\}$. We repeat the same for the face on the opposite side. Considering it is impossible for the angles to be more than half a turn on both sides, we have replaced at least one set of edges and have therefore successfully resolved the degeneracy and flipped the region adjacency.

\begin{figure}[H]
	\centering
	\subfigure[]{\includegraphics[width=30mm]{Resources/FlipInternalEdge-Expand-1.pdf}}
	\quad
	\subfigure[]{\includegraphics[width=30mm]{Resources/FlipInternalEdge-Expand-2.pdf}}
	\quad
	\subfigure[]{\includegraphics[width=30mm]{Resources/FlipInternalEdge-Expand-3.pdf}}
	\caption{A contact representation before (a) and after (c) creating a non-degenerate yellow-blue adjacency. (b) shows that $q_2$ cannot be the midpoint of $p_2$ and $p_5$ and needs to move further towards $p_3$.}
	\label{fig:flip-internal-edge-expand}
\end{figure}
