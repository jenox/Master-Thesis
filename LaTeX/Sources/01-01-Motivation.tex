% "What motivates our concrete problem statement?"
\section{Motivation}
\label{sect:motivation}

% Clustered data
Clustered data appears naturally in many real-world data sets. For example, university students can be grouped by their academic major, or politicians can be grouped according to their party affiliation. Other applications call for use case specific definitions of clusters to manage with the ever-growing amount of data. But no matter where the clusters come from, they add an immense amount of structure to the data and are worth visualizing explicitly.

% Map metaphor
Given a map of the earth, it is remarkably easy to find out of two regions are neighbors, or to compare their size. We are confronted with such tasks starting at a very young age, and absorbing data through this visual channel becomes very natural for many of us. In fact, we generally enjoy working with maps \cite{saket2016comparing}. Skupin and Fabrikant \cite{skupin2003spatialization} realized the relevance of applying a cartographic perspective to general information visualization very early and motivated research of transferring the geographic map metaphor to non-geographic domains. In the following years, map-based graph visualizations, \ie{} drawings of graphs in which the vertices' cluster information is explicitly encoded as colored regions of the plane, have been studied and evaluated in great detail. Many popular graph drawing algorithms such as GMap \cite{gansner2009gmap} or MapSets \cite{efrat2014mapsets} make use of the map metaphor to convey clustering information in a familiar fashion. Saket \etal{} \cite{saket2014group} proposed a set of tasks that assess a viewer's ability to detect clustering information in graph visualizations. They evaluated these tasks across different types of diagrams \cite{saket2014node} and found that map-based diagrams outperform both node-link diagrams on tasks related to detecting clustering information and do not negatively affect the performance of other tasks. Map-based graph visualizations also have greater memorability of data \cite{saket2015map} and are more enjoyable to work with \cite{saket2016comparing} than traditional node-link diagrams. In this thesis, we too shall utilize the map metaphor and encode the clustering information by placing all vertices of a cluster in the same contiguous region of a map.

% Area-proportionality
\todo{- area-proportionality common in cartograms that are well evaluated and understood.}

% Dynamic aspect + mental map
In today's world, it's not only big data that we're dealing with. Data is also changing at a much greater pace than before: visualizations that were accurate a moment ago may be outdated now. For many applications it is therefore crucial to visualize how data evolves over time. Visualizing a dynamic graph brings has own challenges though: We are essentially producing a sequence of snapshots at different points in time. Upon viewing such a snapshot, we inevitably create some internal representation of what we see in our minds. It is crucial that this mental model remains consistent throughout the visualization. If it doesn't, viewers may not be able to see the overall trends in the dynamic data because they need re-orient often \cite{bohringer1990using} \cite{lee2006mental} \cite{purchase2006important}. This fundamental aesthetic criterion of dynamic graph visualizations was coined \emph{preserving the mental map} by Eades \etal{} \cite{eades1991preserving} \cite{misue1995layout} and is also known as \emph{dynamic stability} \cite{diehl2002graphs}. We shall preserve the mental map of the dynamic visualization by maintaining the combinatorial embedding of the map and only allowing small, and breaking changes down into small, incremental pieces.

% Motivating application: conrete use case
A real-world application with great potential to benefit from this framework and its goals is the visualization of \emph{opinion networks} \cite{betz2019applying} and in particular how they evolve over time. An opinion network is represented as a weighted graph whose vertices are so-called \emph{opinion vectors} and the edge weights represent the similarity between two opinion vectors. Betz \etal{} cluster the vertices to group similar opinions together, visualize the graph as a map in which each cluster corresponds to a country, and draw the original graph on top of this map. Naturally, such an opinion network is dynamic in a sense where new opinions can be incorporated or individual opinions can change over time: clusters can therefore grow or shrink, appear or disappear, merge or split. We are interested in visualizing such processes effectively to find and communicate important trends in the underlying data.
