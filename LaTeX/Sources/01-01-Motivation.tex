\section{Motivation}
\label{sect:motivation}

We live in a world with an abundance of data. Data for which it is no longer possible for humans alone to work through, to make sense of, and data that is only going to increase in volume in the years to come. As the amount of data we work with grows, it becomes increasingly important to properly visualize the data in order to gain key understandings and to communicate those with others. It is crucial to make use of this visual perception channel as it has the highest information bandwidth of all our senses. In fact, humans acquire more information through vision alone than through all other senses combined \cite{ware2019information}.

As the amount of data that needs to be visualized keeps growing while the human ability to process data remains constant, it becomes increasingly important to capture the most relevant aspects of the data \cite{dachsbacher2019visualisierung}. In the field of graph theory and graph drawing, for example, for larger graphs, global structures such as clusters and their relationships become increasingly important while local features such as individual edges loose importance.

\new{
Saket \etal{} \cite{saket2014group} propose a set of tasks that assess a user's ability to detect clustering information in graph visualizations. They evaluated these tasks across different types of diagrams \cite{saket2014node} and found that node-link-group diagrams, \ie{} diagrams in which there is an explicit representation of clusters, outperform both node and node-link diagrams on group-based tasks and do not negatively affect the performance of other tasks. According to another study by Saket \etal{} \cite{saket2016comparing}, node-link-group diagrams are also more enjoyable for users to work with. In this thesis, we make use of the map metaphor and encode the clustering information by placing all vertices of a cluster in the same \quoted{country} of a map.
}

But in today's world it's not only big data that we're dealing with. Data is also changing at a much greater pace than before. Visualizations that were accurate a moment ago may be outdated now. For many applications it is therefore crucial to visualize how data evolves over time \emdash{} maybe even more so than a snapshot of the data at a single point in time.

A core motivating application for this thesis is visualizing an \emph{opinion network} \cite{betz2019applying} and in particular visualizing how it evolves over time: An opinion network is represented as a weighted graph whose vertices are so-called \emph{opinion vectors} and the edge weights represent the similarity between two opinion vectors. Betz \etal{} cluster the vertices to group similar opinions together, visualize the graph as a map in which each cluster corresponds to a country, and draw the original graph on top of this map. Naturally, such an opinion network is dynamic in a sense where new opinions can be incorporated or individual opinions can change over time. Therefore clusters can grow or shrink, appear or disappear, merge or split, and we are interested in visualizing this underlying process over time.
