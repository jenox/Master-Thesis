\section{Quality Metrics}
\label{sect:quality-metrics}

We start by looking at common quality metrics of cartograms, which are closely related to the problem we are trying to solve in this thesis as previously discussed in \cref{sect:related-work}, and discussing how they translate to the visualizations our framework creates.
We then discuss different ideas to formalize and quantify the previously mentioned notion of organicness and local fatness of the regions.



\paragraph{Quality Metrics of Cartograms}

Three well-established quantifiable measures are commonly used to assess a cartograms's quality \cite{alam2015quantitative} \cite{nusrat2018evaluating}:

% TODO: actually define cartographic error first! probably in preliminaries.

\begin{itemize}
\item \textbf{Statistical accuracy:}
The statistical accuracy of a cartogram describes how closely the areas of the modified geographic regions match the variable of interest.
Recall that the cartographic error of a region $v$ is defined as $\frac{\abs{A(v)-w(v)}}{max(A(v),w(v))}$ where $A(v)$ is its actual area and $w(v)$ is its desired area.
The maximum and average cartographic error over all regions in the cartogram are commonly used to quantify its statistical accuracy.
We borrow this quality metric from cartograms and use it for our visualizations in the form of $\varepsilon$-area-proportionality.

\item \textbf{Topological accuracy:}
The topological accuracy describes how well the original adjacencies between the geographic regions are preserved in the cartogram.
Considering our framework computes a polygonal contact representation of the filtered cluster graph, those and only those regions whose corresponding vertices are adjacent in the cluster graph are adjacent in the contact representation.
In terms of topological accuracy, we therefore produce perfect drawings.

\item \textbf{Geographic accuracy:}
The geographic accuracy of a cartogram describes to what degree the shapes and positions of the distorted regions resemble their original counterpart on the real geographic map.
In our case, however, there is no real geographic map that our visualization is based upon:
The maps we generate are entirely artificial and there is no geographic reference map.

But why is this resemblance a meaningful quality metric?
Geographic accuracy captures the preservation of the viewer's mental map between the real geographic map and the distorted map in the cartogram.
For our framework, this is only relevant once we start incorporating dynamic updates into the artificial map.
Our dynamic pipeline naturally preserves the viewer's mental map by only allowing small, incremental changes to be incorporated and redrawing the artificial map using a force-directed algorithm.
This makes it easy for the viewer to track how the map changes between versions of the underlying cluster graph at different points in time.
\end{itemize}



\paragraph{Organicness and Local Fatness}

\todo{fraction or convex hull \cite{brinkhoff1995measuring}}

\todo{fraction of circumcircle}

\todo{incircle / circumcircle}

\todo{entropy of angles \cite{chen2005estimating}}

\todo{entropy of distances from centroid \cite{chen2005estimating}}

which of these captures the local fatness and organicness best? artificial examples. pick 2 to proceed with in \cref{sect:evaluation}.