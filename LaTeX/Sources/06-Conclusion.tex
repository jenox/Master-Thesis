\chapter{Conclusion}
\label{chap:conclusion}

\begin{itemize}
	\item relax requirements on cluster graph, e.g. allow countries on outer face that are adjacent to only one country (found in application and on natural maps)
	\item implement more dynamic operations
	\item improve implementation of individual phases / black boxes \begin{itemize}
		\item improve initial layout: can start out with perfect statistical accuracy with just one bend per edge (Kleist) (always!) or even without bends at all (thomassen) if we don't relax requirements too much
		\item improve on forces or use different optimization method altogether
		\item improve on precise mathematical formulation of shapes we are looking for. and optimization working towards that.
	\end{itemize}
	\item evaluate how our quality metrics relate to human perception and the map metaphor
	\item evaluate how quality metrics differ when applying incremental changes vs starting from scratch. this is where "geographical accuracy" from cartograms comes into play.
\end{itemize}


%\todo{incircle / circumcircle} in future work
