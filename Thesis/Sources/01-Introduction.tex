\chapter{Introduction}
\label{chap:introduction}

We live in a world with an abundance of data.
Data for which it is no longer possible for humans alone to work through, or to make sense of, and data that is only going to increase in volume in the years to come.
As the amount of data we work with soars to previously uncharted heights and analyzing it seems increasingly infeasible, visualizations retain their ability to help discover useful information, to illustrate relationships in large sets of data, and to communicate them with others effectively.

Even in today's highly automated world, many processes still require some degree of human decision-making and interaction.
Data visualizations are a great tool to inform these decisions as the visual perception channel has the highest information bandwidth of all human senses.
In fact, we acquire more information through vision alone than through all other senses combined \cite{ware2019information}.
However, as the mountains of data that we seek to understand keep growing, the human ability to process data remains mostly constant \cite{dachsbacher2019visualisierung}.
Therefore, it becomes increasingly important to capture the most relevant aspects of the data and its trends with computer-aided visualizations.

% What motivates our concrete problem statement?
\section{Motivation}
\label{sect:motivation}

% Clustered data
Clustered data appears naturally in many real-world data sets. For example, university students can be grouped by their academic major, or politicians can be grouped according to their party affiliation. When done correctly, clustering provides a lot of additional structure to data and essentially functions as a high-level data aggregation that is easy to visualize while still conveying crucial global information.
In this thesis, we therefore propose a framework that explicitly visualizes clusters in the input graph as a high-level overview in addition to all local details.

% Map metaphor
Given a map of the earth, it is remarkably easy to find out whether or not two countries are neighbors, or to compare their size. We are confronted with such tasks starting at a very young age, and absorbing data through this visual channel becomes very natural for most of us. In fact, we generally enjoy working with maps \cite{saket2016comparing}. Skupin and Fabrikant \cite{skupin2003spatialization} realized the relevance of applying a cartographic perspective to general information visualization very early and motivated research of transferring the geographic map metaphor to non-geographic domains. In the following years, map-based graph visualizations, \ie{} drawings of graphs in which the vertices' cluster information is explicitly encoded as colored regions of the plane, have been studied and evaluated in great detail. Such visualizations outperform traditional node-link diagrams on tasks related to detecting clustering information while not negatively affect the performance of other tasks \cite{saket2014node}, provide greater memorability of data \cite{saket2015map}, and are more enjoyable to work with \cite{saket2016comparing}.
In this thesis, we too shall utilize the map metaphor and encode the clustering information by placing all vertices of a cluster in a contiguous region on the map.

% Dynamic aspect + mental map
In today's world, it's not only big data that we're dealing with. Data is also changing at a much greater pace than before: visualizations that were accurate yesterday may be outdated today. Many applications deal with volatile data by nature, such as stock prices or, in light of current events, the number of coronavirus infections. It is therefore crucial to not only visualize the data at a single point in time, but to also visualize trends in the data as it evolves over time. But visualizing a dynamic graph brings its own challenges: We are essentially producing a sequence of snapshots at different points in time. Upon viewing such a snapshot, we inevitably create some internal representation of what we see in our minds. It is crucial that this mental model remains consistent throughout the visualization: If it doesn't, viewers may not be able to see the overall trends in the dynamic data because they need re-orient often \cite{bohringer1990using} \cite{lee2006mental} \cite{purchase2006important}. This fundamental aesthetic criterion of dynamic graph visualizations was coined \emph{preserving the mental map} by Eades \etal{} \cite{eades1991preserving} \cite{misue1995layout} and is also known as \emph{dynamic stability} \cite{diehl2002graphs}.
In this thesis, we shall preserve the mental map of the dynamic map visualization by maintaining the combinatorial embedding of the map and breaking changes down into small, incremental pieces that can easily be applied to the dynamic map.

% Area-proportionality
Cartograms are maps in which geographic regions appear skewed such that their areas are proportional to some statistic, \eg{} the population or gross domestic product of a country. Although cartograms are traditionally used to visualize demographic data and are based real geographic maps, extensive studies related to human perception \cite{nusrat2016state} \cite{nusrat2018evaluating} give us a few valuable insights that can be transferred to our problem definition: First, area is a strong visual variable that can be interpreted naturally. And second, having before/after-comparison allows viewers to easily detect trends in the underlying data.
In this thesis, we therefore aim to create maps whose regions' areas are proportional to their respective cluster sizes and adopt established quality metrics of cartograms for the visualizations we produce.

% Motivating application: concrete use case
A real-world application with great potential to benefit from this framework and its goals is the visualization of \emph{opinion networks} \cite{betz2019applying} and in particular how they evolve over time. An opinion network is represented as a weighted graph whose vertices are so-called \emph{opinion vectors} and the edge weights represent the similarity between two opinion vectors. Betz \etal{} \cite{betz2019applying} cluster the vertices to group similar opinions together, visualize the graph as a map in which each cluster corresponds to a country, and draw the original graph on top of this map. Naturally, such an opinion network is dynamic in a sense where new opinions can be incorporated or individual opinions can change over time: clusters can therefore grow or shrink, appear or disappear, merge or split. We are interested in visualizing such processes effectively in order to find and communicate important trends in the underlying data.

% How are we different from concrete other research?
\section{Related Work}
\label{sect:related-work}

\cite{lee2006mental}
\cite{lewandowsky1993perception}
\cite{kobourov2012putting}
\cite{purchase2008extremes}
\cite{purchase2006important}
\cite{saket2015map}
\cite{efrat2014mapsets}

multiple paragraphs? (like MapSets)

mapsets?
gmap? -> fragmented.
OpMAP?

%three core components/requirements/objectives:
%- area-proportional maps
%- dynamic aspect
%- natural shapes

% skewed, distorted, deformed
% variable of interest, statistic

While a vertex-weighted cluster graph is the intuitive representation of the data we want to visualize, we'll end up doing most of the work in the following chapters on the dual problem: We transform the vertex-weighted cluster graph into its dual graph. The weights then apply to the faces of dual graphs and we are looking to draw the dual such that the faces have some prescribed area. A drawing of the dual essentially is a \emph{contact representation} of the original graph: iff two faces share part of their boundaries, the corresponding vertices in the original graph are adjacent.

\emph{Area-universal graphs} are graphs that can realize any area assignment to its inner faces with straight-line edges. Research on area-universality give us important theoretical bounds on the statistical accuracy we can achieve with polygonal countries: Back in 1992, Thomassen \cite{thomassen1992plane} showed that plane cubic graphs are area-universal. This means that we can achieve perfect statistical accuracy with straight-line edges for the dual of any triangulated graph. Kleist \cite{kleist2018drawing} \cite{kleist2019planar} showed that the 1-subdivision of any plane graph is area-universal. Therefore with just one bend per edge, any plane graph can be drawn with any prescribed face areas. Biedl and Velázquez \cite{biedl2013drawing} showed that the class of 3-trees and subgraphs thereof are area-universal, providing a constructive proof based on barycentric coordinates.

Drawing graphs with prescribed face areas is also closely related to so-called \emph{cartograms}. Cartograms are maps in which geographic regions appear skewed such that their areas are proportional to some statistic, \eg{} the population or gross domestic product of a country. Cartograms have been studied for more than 50 years \cite{tobler2004thirty} and there are many fundamentally different approaches to generate different kinds of cartograms \cite{nusrat2016state}. The effectiveness of cartograms related to human perception has been evaluated extensively \cite{nusrat2018evaluating} and the following three quantitative metrics are most commonly used to judge the quality of cartograms \cite{nusrat2016state} \cite{alam2015quantitative} \cite{nusrat2018evaluating}:
%
\begin{itemize}
	\item \emph{Statistical accuracy}: How closely do the modified geographic areas represent the variable of interest?
	\item \emph{Topological accuracy}: How well is the map topology, \ie{} the adjacency relationship between geographic regions, preserved?
	\item \emph{Geographic accuracy}: How closely do the distorted geographic shapes and positions resemble their original?
\end{itemize}

Although cartograms are traditionally used to visualize demographic data on real geographic maps, they are closely related to the problem at hand: visualizing clusters such that their areas are proportional to their sizes, \ie{} the number of vertices. Statistical and topological accuracy still apply, but geographical accuracy becomes a meaningless quality metric because the visualization isn't based on any natural geographic map and there are no \quoted{original} versions of the country shapes.

Nusrat and Kobourov \cite{nusrat2016state} give an overview of many different algorithms for generating cartograms and discuss how they stack up against one another.

Gastner and Newman \cite{gastner2004diffusion} propose a physical model based on diffusion to generate cartograms: They rasterize the original map into a two-dimensional matrix with the values being the initial densities, \ie{} the statistical values divided by the regions' areas at any given point. This matrix is then used to precompute the gradient of the diffusion field and the pathlines of these \quoted{density particles} as they diffuse through the map and equalize the overall density. The pathlines essentially map locations on the original map to their location in the diffused map and can be used to draw the distorted, density-equalizing map. Due to the rasterization and heavy precomputation of pathlines, this algorithm isn't well-suited for our dynamic setting in which densities can change.

Kämper \etal{} \cite{kamper2013circular} start with a polygonal map and transform every edge into a circular arc that can bend to realize the desired areas of individual regions. They use a max-flow-based formulation on the dual graph of the map to find out how the area should be distributed among the regions and solve for the circular arc radii. However the degree to which the edges can bend is heavily restricted since the circular arcs may not touch or cross, making it difficult for circular arc cartograms to achieve good statistical accuracies.

Alam \etal{} \cite{alam2013computing} show how air-pressure-based models for the general floorplan problem such as \cite{izumi1998air} and \cite{felsner2013exploiting} can be applied to generating cartograms. Each region is assigned a target area based on the statistic we want to visualize. One can then compute the pressure in each region based on its current area and target area at different steps in the algorithm and use it to iteratively grow and/or shrink the regions until the target areas are achieved. These ideas motivate the force-directed formulation of our problem in a later chapter.

GMap \cite{gansner2009gmap} is an algorithm that visualizes graphs as geographical maps. In a first step, it embeds the graph in the plane using a traditional force-directed graph drawing algorithm. The embedded vertices are then clustered using a geometric clustering algorithms such as k-means. Using the initial drawing and the clustering, a map is created based on a Voronoi diagram of the vertices. By drawing the input graph first and then clustering it based on the drawing, we potentially lose out on structure in the abstract graph that wasn't captured by the drawing, that an abstract graph clustering algorithm may have picked up. In this thesis we therefore cluster the graph and extract important features before embedding the graph in the plane. In a follow-up paper, Mashima \etal{} \cite{mashima2011visualizing} build upon GMap to visualize dynamic input graphs while maintaining the mental map of a viewer.

\section{Contribution of this Thesis}
\label{sect:structure-of-this-thesis}

The main problem with GMap is the independent nature of the clustering and embedding phases:
If clustering is done first, a force-directed embedding algorithm likely causes a high degree of fragmented and noncontinuous regions \cite{mashima2011visualizing}.
OpMap manages to somewhat diminish this effect with a smart choice of forces that makes use of the clustering information \cite{schmettow2017}.
Embedding the graph first and then clustering it using a geometric clustering algorithm is dangerous, too, because this may detect clusters that do not exist in the abstract data.
Most force-directed algorithms used with GMap also lack the ability to specify the desired sizes and shapes of the clusters.

OpMap, as implemented in \cite{schmettow2017}, supports only some degree of dynamicity: while new vertices can be added to existing clusters dynamically, the set of clusters is determined statically.
This means that as new opinions are incorporated into the opinion network, no new clusters can form.

In this thesis, we address the aforementioned issues in related work and try to overcome them.
In particular, we design an algorithm that constructs maps in which the countries are guaranteed to be continuous, whose areas are close to proportional to the respective cluster sizes, and whose shapes are somewhat organic to resemble real-world maps.
The algorithm also allows for dynamic updates such as inserting new clusters, removing existing clusters, tweaking cluster adjacencies, or changing cluster weights.
To evaluate our algorithm, we adopt several measures from literature, capturing both the map's accuracy and the quality of its regions' shapes.
Finally, we experimentally evaluate the quality of the maps created by our algorithm using randomly generated data sets.

\section{Structure of this Thesis}
\label{sect:structure-of-this-thesis}

In \cref{chap:preliminaries} we present preliminary definitions of various concepts that we base our algorithms and discussions on, some of which were already brought up earlier in this chapter without formal definitions.
We then define our algorithmic pipeline for static inputs in \cref{chap:visualizing-static-input-graphs} and for dynamic inputs in \cref{chap:visualizing-dynamic-input-graphs} and discuss possible implementations of the individual phases of the pipeline.
In \cref{chap:experimental-evaluation} we conduct an evaluation of the visualizations produced by our implementations on a number of test inputs.
Eventually, in \cref{chap:conclusion}, we discuss shortcomings of our framework and how future research can improve on it.

