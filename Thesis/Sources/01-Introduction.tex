\chapter{Introduction}
\label{chap:introduction}

We live in a world with an abundance of data.
Data for which it is no longer possible for humans alone to work through, or to make sense of, and data that is only going to increase in volume in the years to come.
As the amount of data we work with soars to previously uncharted heights and analyzing it seems increasingly infeasible, visualizations retain their ability to help discover useful information, illustrate relationships in large sets of data, and efficiently communicate them with others.

Even in today's highly automated world, many processes still require some degree of human decision-making and interaction.
Data visualizations are a great tool to inform these decisions as the visual perception channel has the highest information bandwidth of all human senses.
In fact, we acquire more information through vision alone than through all other senses combined \cite{ware2019information}.
However, as the mountains of data that we seek to understand keep growing, the human ability to process data remains mostly constant \cite{dachsbacher2019visualisierung}.
Therefore, it becomes increasingly important to capture the most relevant aspects of the data and its trends with computer-aided visualizations.

% What motivates our concrete problem statement?
\section{Motivation}
\label{sect:motivation}

% Clustered data
Clustered data appears naturally in many real-world data sets. For example, university students can be grouped by their academic major, or politicians can be grouped according to their party affiliation. When done correctly, clustering provides a lot of additional structure to data and essentially functions as a high-level data aggregation that is easy to visualize while still conveying crucial global information.
In this thesis, we therefore propose a framework that explicitly visualizes clusters in the input graph as a high-level overview in addition to all local details.

% Map metaphor
Given a map of the earth, it is remarkably easy to find out whether or not two countries are neighbors, or to compare their size. We are confronted with such tasks starting at a very young age, and absorbing data through this visual channel becomes very natural for most of us. In fact, we generally enjoy working with maps \cite{saket2016comparing}. Skupin and Fabrikant \cite{skupin2003spatialization} realized the relevance of applying a cartographic perspective to general information visualization very early and motivated research of transferring the geographic map metaphor to non-geographic domains. In the following years, map-based graph visualizations, \ie{} drawings of graphs in which the vertices' cluster information is explicitly encoded as colored regions of the plane, have been studied and evaluated in great detail. Such visualizations outperform traditional node-link diagrams on tasks related to detecting clustering information while not negatively affect the performance of other tasks \cite{saket2014node}, provide greater memorability of data \cite{saket2015map}, and are more enjoyable to work with \cite{saket2016comparing}.
In this thesis, we too shall utilize the map metaphor and encode the clustering information by placing all vertices of a cluster in a contiguous region on the map.

% Dynamic aspect + mental map
In today's world, it's not only big data that we're dealing with. Data is also changing at a much greater pace than before: visualizations that were accurate yesterday may be outdated today. Many applications deal with volatile data by nature, such as stock prices or, in light of current events, the number of coronavirus infections. It is therefore crucial to not only visualize the data at a single point in time, but to also visualize trends in the data as it evolves over time. But visualizing a dynamic graph brings its own challenges: We are essentially producing a sequence of snapshots at different points in time. Upon viewing such a snapshot, we inevitably create some internal representation of what we see in our minds. It is crucial that this mental model remains consistent throughout the visualization: If it doesn't, viewers may not be able to see the overall trends in the dynamic data because they need re-orient often \cite{bohringer1990using} \cite{lee2006mental} \cite{purchase2006important}. This fundamental aesthetic criterion of dynamic graph visualizations was coined \emph{preserving the mental map} by Eades \etal{} \cite{eades1991preserving} \cite{misue1995layout} and is also known as \emph{dynamic stability} \cite{diehl2002graphs}.
In this thesis, we shall preserve the mental map of the dynamic map visualization by maintaining the combinatorial embedding and outer face of the map and breaking changes down into small, incremental pieces that can easily be applied to the dynamic map.

% Area-proportionality
Cartograms are maps in which geographic regions appear skewed such that their areas are proportional to some statistic, \eg{} the population or gross domestic product of a country. Although cartograms are traditionally used to visualize demographic data and are based real geographic maps, extensive studies related to human perception \cite{nusrat2016state} \cite{nusrat2018evaluating} give us a few valuable insights that can be transferred to our problem definition: First, area is a strong visual variable that can be interpreted naturally. And second, having before/after-comparison allows viewers to easily detect trends in the underlying data.
In this thesis, we therefore aim to create maps whose regions' areas are proportional to their respective cluster sizes and adopt established quality metrics of cartograms for the visualizations we produce.

% Motivating application: concrete use case
A real-world application with great potential to benefit from this framework and its goals is the visualization of \emph{opinion networks} \cite{betz2019applying} and in particular how they evolve over time. An opinion network is represented as a weighted graph whose vertices are so-called \emph{opinion vectors} and the edge weights represent the similarity between two opinion vectors. Betz \etal{} \cite{betz2019applying} cluster the vertices to group similar opinions together, visualize the graph as a map in which each cluster corresponds to a country, and draw the original graph on top of this map. Naturally, such an opinion network is dynamic in a sense where new opinions can be incorporated or individual opinions can change over time: clusters can therefore grow or shrink, appear or disappear, merge or split. We are interested in visualizing such processes effectively in order to find and communicate important trends in the underlying data.

% How are we different from concrete other research?
\section{Related Work}
\label{sect:related-work}


\paragraph{GMap and OpMap}

GMap \cite{gansner2009gmap} is a framework for visualizing graphs as geographical maps.
In a first step, it embeds the input graph in the plane and performs a cluster analysis using arbitrary user-provided algorithms.
Embedding and clustering are typically done independently from one another; however, the algorithms should be well-matched such that clusters form continuous regions in the computed embedding.
A combination of force-directed graph drawing and modularity-based clustering methods is commonly used.
GMap then creates a map by computing the Voronoi diagram of the graph's vertices as per the computed embedding and merging adjacent cells whose vertices belong to the same cluster.
Eventually, the original graph's edges are drawn on top of this map.

In a follow-up paper, Mashima \etal{} \cite{mashima2011visualizing} built upon GMap to visualize dynamic input graphs.
To maintain the mental map of the viewer, they create a \quoted{canonical map} storing positional information of a much larger graph than the one that is eventually going to be shown to the viewer.
At different points in time, only the most prominent vertices and clusters of the canonical map are then visualized.
Because the canonical map is much less volatile, vertices and clusters don't move as much on the resulting map.

The aforementioned algorithm for visualizing opinion networks by Betz \etal{} is called OpMap.
OpMap focuses on clustering and weighting the opinion vectors in the network and delegates the visualization to GMap.
However, the clustering is performed on the abstract input graph, independent from the force-directed embedding algorithm.
This can be problematic, because if the clustering and the computed embedding don't fit together well, one ends up with many fragmented, noncontinuous clusters on the map.
OpMap has only been evaluated for relatively small input sizes and works with static inputs only.
We want to explore an approach that supports larger \emdash{} and most importantly dynamic \emdash{} opinion networks.
%We will not attempt to answer open questions from the OpMap paper such as how to scale the clustering to larger graphs, or how adapt the clustering to a dynamic setting.

In this thesis we therefore cluster the graph and extract important features first, and only then embed the graph in the plane while keeping the clusters connected.


\paragraph{Cartograms}

Cartograms have been studied for more than 50 years \cite{tobler2004thirty} and there are lots of fundamentally different approaches to generate different kinds of cartograms \cite{nusrat2016state}.
Their quality is generally assessed by a combination of statistical accuracy, topological accuracy, and geographic accuracy \cite{alam2015quantitative}: \emph{Statistical accuracy} describes how closely the areas of the modified geographic regions match the variable of interest, \emph{topological accuracy} to what degree the original region adjacencies are preserved, and \emph{geographic accuracy} how well the shapes and positions of the distorted regions resemble their original.
Geographic accuracy is closely related to the preservation of the mental map as it captures a viewer's ability to intuitively recognize the regions and therefore detect the trends in the underlying data.
Although traditionally used for other applications, cartograms are closely related to our problem statement and the aforementioned quality metrics translate directly to our use case.
While we don't have a reference map for the initial area-proportional map like cartograms do, we can use that initial map as a reference map that we can base geographical accuracy on when incorporating dynamic updates of the input graph.
There is a large body of literature on the generation of cartograms \cite{tobler2004thirty} \cite{alam2015quantitative} \cite{nusrat2016state}.
Here, we mention only the most relevant techniques and explain to what degree we have adopted them.

Gastner and Newman \cite{gastner2004diffusion} propose a physical model based on diffusion to generate cartograms:
They rasterize the original map into a two-dimensional matrix with the values being the initial densities, \ie{} the statistical values divided by the regions' areas at any given point.
This matrix is then used to precompute the gradient of the diffusion field and the pathlines of these \quoted{density particles} as they diffuse through the map and equalize the overall density.
The pathlines essentially map locations on the original map to their location in the diffused map and can be used to draw the distorted, density-equalizing map.
However, due to the rasterization and heavy precomputation of pathlines, this algorithm isn't well-suited for our dynamic setting in which densities can change and are not necessarily known a priori.

Kämper \etal{} \cite{kamper2013circular} start with a polygonal map and transform every edge into a circular arc that can bend to realize the desired areas of individual regions.
They use a max-flow-based formulation on the dual graph of the map to find out how the area should be distributed among the regions and solve for the circular arc radii.
However, the degree to which the edges can bend is heavily restricted since the circular arcs may not touch or cross, making it difficult for circular arc cartograms to achieve good statistical accuracy.
The resulting regions' shapes also appear very artificial, unlike the ones you would find on a real geographic map.

Alam \etal{} \cite{alam2013computing} show how air-pressure-based models for the general floorplan problem such as \cite{izumi1998air} and \cite{felsner2013exploiting} can be applied to generating rectilinear cartograms.
They give a force-directed heuristic to iteratively compute the cartogram and experimentally show very fast convergence to more than 99\% accuracy.
Each region is assigned a target area based on the statistic one wants to visualize.
They then compute the pressure in each region based on its current area and target area and use it to iteratively grow and/or shrink the regions by shifting its borders.
This heuristic motivates the force-directed formulation of a pivotal part of our framework whose theoretical background we discuss in the following paragraph.


\paragraph{Force-directed Graph Drawing}

Force-directed graph drawing algorithms regard the graph to be visualized as a physical system in which the vertices are individual particles, and several forces are acting on said particles.
These forces are defined such that they act to bring the system into a stable equilibrium position in which its potential energy is at a local minimum and the resulting drawing has certain desired features.
Eades \cite{eades84heuristic} first used a combination of attractive forces between adjacent vertices based on physical springs and repulsive forces between all pairs of vertices based on electric repulsion.
These forces result in adjacent vertices being pulled together while non-adjacent vertices are pushed further apart from each other.
The drawings resulting from force-directed algorithms are generally visually appealing and easy to grasp \cite{kobourov2013force}.

Bertault \cite{bertault1999force} designed a force-directed algorithm called PrEd that imposes additional constraints on the displacement of vertices.
In PrEd vertices are only allowed to move in a way that preserves the edge crossing properties of the initial layout, \ie{} existing edge crossings are preserved and no new edge crossings are introduced.
Simonetto \etal{} \cite{simonetto2011impred} introduced an improved version of this algorithm called ImPrEd that is more flexible and performs much better on larger input graphs.

In this thesis, we use a force-directed algorithm with custom forces to control the layout of our polygonal maps.
The forces we use are based on the aforementioned \quoted{pressure} in the individual regions of the map.
It is crucial that we do not introduce edge crossings when tweaking the map's layout because regions aren't allowed to overlap on geographical maps.


\paragraph{Area Universality}

Area-universal graphs are plane graphs that can realize any area assignment to its internal faces with straight-line edges.
Research on area-universality gives us important theoretical bounds on the statistical accuracy we can achieve with polygonal maps.

Back in 1992, Thomassen \cite{thomassen1992plane} showed that plane cubic graphs are area-universal.
For polygonal maps this means that, as long as at most three regions meet in a point, we can achieve perfect statistical accuracy for arbitrary region weights.
This is because if we were to eliminate all corners that are part of only two region boundaries, all corners in the resulting polygonal map would would be are part of exactly three region boundaries, and the map could therefore be interpreted as a plane cubic graph.

Kleist \cite{kleist2018drawing} \cite{kleist2019planar} showed that the 1-subdivision of any plane graph is area-universal.
With just one bend per edge, any plane graph can therefore be drawn with arbitrary prescribed face areas.
This, too, translates to polygonal maps: as long as there's a point that's part of at most two region boundaries between two points that are part of more than two region boundaries, we can lift the requirement of no more than three regions meeting in a point, all while still being able to achieve perfect statistical accuracy for arbitrary region weights.

\section{Contribution of this Thesis}
\label{sect:structure-of-this-thesis}

The main problem with GMap is the independent nature of the clustering and embedding phases:
If clustering is done first, a force-directed embedding algorithm likely causes a high degree of fragmented and noncontinuous regions \cite{mashima2011visualizing}.
OpMap manages to somewhat diminish this effect with a smart choice of forces that makes use of the clustering information \cite{schmettow2017}.
Embedding the graph first and then clustering it using a geometric clustering algorithm is dangerous, too, because this may detect clusters that do not exist in the abstract data.
Most force-directed algorithms used with GMap also lack the ability to specify the desired sizes and shapes of the clusters.

OpMap, as implemented in \cite{schmettow2017}, supports only some degree of dynamicity: while new vertices can be added to existing clusters dynamically, the set of clusters is determined statically.
This means that as new opinions are incorporated into the opinion network, no new clusters can form.

In this thesis, we address the aforementioned issues in related work and try to overcome them.
In particular, we design an algorithm that constructs maps in which the countries are guaranteed to be continuous, whose areas are close to proportional to the respective cluster sizes, and whose shapes are somewhat organic to resemble real-world maps.
The algorithm also allows for dynamic updates such as inserting new clusters, removing existing clusters, tweaking cluster adjacencies, or changing cluster weights.
To evaluate our algorithm, we adopt several measures from literature, capturing both the map's accuracy and the quality of its regions' shapes.
Finally, we experimentally evaluate the quality of the maps created by our algorithm using randomly generated data sets.

\section{Structure of this Thesis}
\label{sect:structure-of-this-thesis}

In \cref{chap:preliminaries} we present preliminary definitions of various concepts that we base our algorithms and discussions on, some of which were already brought up earlier in this chapter without formal definitions.
We then define our algorithmic pipeline for static inputs in \cref{chap:visualizing-static-input-graphs} and for dynamic inputs in \cref{chap:visualizing-dynamic-input-graphs} and discuss possible implementations of the individual phases of the pipeline.
In \cref{chap:experimental-evaluation} we discuss different quantifiable measures to assess the quality of the visualizations produced by our implementation and apply those in an experimental evaluation on a number of test inputs.
Eventually, in \cref{chap:conclusion}, we discuss shortcomings of our framework and explore how future research can improve upon it.

