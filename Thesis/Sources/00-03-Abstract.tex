\thispagestyle{plain}

% https://www.easterbrook.ca/steve/2010/01/how-to-write-a-scientific-abstract-in-six-easy-steps/
% The first sentence of an abstract should clearly introduce the topic of the paper so that readers can relate it to other work they are familiar with.
% However, an analysis of abstracts across a range of fields show that few follow this advice, nor do they take the opportunity to summarize previous work in their second sentence.
% A central issue is the lack of structure in standard advice on abstract writing, so most authors don’t realize the third sentence should point out the deficiencies of this existing research.
% To solve this problem, we describe a technique that structures the entire abstract around a set of six sentences, each of which has a specific role, so that by the end of the first four sentences you have introduced the idea fully.
% This structure then allows you to use the fifth sentence to elaborate a little on the research, explain how it works, and talk about the various ways that you have applied it, for example to teach generations of new graduate students how to write clearly.
% This technique is helpful because it clarifies your thinking and leads to a final sentence that summarizes why your research matters.

\begin{addmargin}{0.5cm}

\centerline{\textbf{Abstract}}

In this thesis, we explore how cluster information in graphs can be visualized naturally as countries on a map in a dynamic setting where the graph changes over time.
We want there to be a strong correlation between a country's size and the size of the cluster it represents, and want to be able to react to dynamic changes of the graph in a way that preserves the viewer's mental model of the map.
This is a challenging problem because popular algorithms for visualizing graphs as geographic-like maps struggle with clusters being fragmented across different countries on the map, and because redrawing the map for a new, albeit similar, input graph does generally not preserve the viewer's mental map.

To address these issues, we propose a framework that guarantees to keep clusters as continuous regions in the map and supports dynamic inputs by allowing for small, incremental updates such as inserting or removing regions over time.
We do this by working on an intermediate plane graph whose vertices correspond to clusters in the input graph, the cluster graph, and constructing the map as a contact representation of this cluster graph while accounting for the dynamic changes in both the cluster graph and its contact representation.
This framework can be applied to a variety of real-world applications, allowing us to visualize clusters in the underlying data and how they change over time, thereby enabling viewers of the visualization to easily detect trends in the data.

\vskip 2cm

\centerline{\textbf{Deutsche Zusammenfassung}}

\selectlanguage{ngerman}
In der vorliegenden Arbeit untersuchen wir, wie Clusterinformationen in Graphen als Länder einer Landkarte dargestellt werden können, wobei sich der Eingabegraph mit der Zeit verändert.
Unsere Ziele sind hierbei eine starke Korrelation zwischen der Größe eines Landes auf der Karte und der Größe des entsprechenden Clusters, und, dass wir so auf Änderungen des Eingabegraphen reagieren können, dass ein Betrachter seine Orientierung auf der Karte nicht verliert.
Das ist eine herausfordernde Problemstellung, da beliebte Algorithmen, um Graphen landkartenähnlich zu visualisieren, mit Clustern zu kämpfen haben, die über mehrere Länder auf der Karte fragmentiert sind.
Außerdem geht die Orientierung des Betrachters bei einer Neuzeichnung der Karte für einen anderen, wennauch sehr ähnlichen Graphen, im Allgemeinen verloren.

Zur Lösung dieser Probleme schlagen wir ein Rahmenkonzept vor, welches garantiert dass die Cluster jeweils zusammenhängende Regionen auf der Karte bilden, und dynamische Eingaben insofern unterstützt, als dass kleine, inkrementelle Änderungen, wie z.B. das Einfügen oder Entfernen einer Region, durchgeführt werden können.
Dafür arbeiten wir auf einem Hilfsgraphen dessen Knoten den Clustern im Eingabegraph entsprechen, konstruieren die Karte als Kontaktrepräsentation dieses Graphen und passen bei Änderungen am Eingabegraphen sowohl den Hilfsgraphen als auch dessen Kontraktrepräsentation entsprechend an.
Dieser Ansatz kann für eine Vielzahl von Anwendungen verwendet werden, um Cluster und deren zeitliche Enwicklung auf eine natürliche Art und Weise darzustellen, und es dem Betrachter somit zu erleichtern, Trends in den zugrundeliegenden Daten zu erkennen.
\selectlanguage{american}

\end{addmargin}
